\documentclass[]{article}
\usepackage{lmodern}
\usepackage{amssymb,amsmath}
\usepackage{ifxetex,ifluatex}
\usepackage{fixltx2e} % provides \textsubscript
\ifnum 0\ifxetex 1\fi\ifluatex 1\fi=0 % if pdftex
  \usepackage[T1]{fontenc}
  \usepackage[utf8]{inputenc}
\else % if luatex or xelatex
  \ifxetex
    \usepackage{mathspec}
  \else
    \usepackage{fontspec}
  \fi
  \defaultfontfeatures{Ligatures=TeX,Scale=MatchLowercase}
\fi
% use upquote if available, for straight quotes in verbatim environments
\IfFileExists{upquote.sty}{\usepackage{upquote}}{}
% use microtype if available
\IfFileExists{microtype.sty}{%
\usepackage{microtype}
\UseMicrotypeSet[protrusion]{basicmath} % disable protrusion for tt fonts
}{}
\usepackage[margin=1in]{geometry}
\usepackage{hyperref}
\hypersetup{unicode=true,
            pdfborder={0 0 0},
            breaklinks=true}
\urlstyle{same}  % don't use monospace font for urls
\usepackage{color}
\usepackage{fancyvrb}
\newcommand{\VerbBar}{|}
\newcommand{\VERB}{\Verb[commandchars=\\\{\}]}
\DefineVerbatimEnvironment{Highlighting}{Verbatim}{commandchars=\\\{\}}
% Add ',fontsize=\small' for more characters per line
\usepackage{framed}
\definecolor{shadecolor}{RGB}{248,248,248}
\newenvironment{Shaded}{\begin{snugshade}}{\end{snugshade}}
\newcommand{\AlertTok}[1]{\textcolor[rgb]{0.94,0.16,0.16}{#1}}
\newcommand{\AnnotationTok}[1]{\textcolor[rgb]{0.56,0.35,0.01}{\textbf{\textit{#1}}}}
\newcommand{\AttributeTok}[1]{\textcolor[rgb]{0.77,0.63,0.00}{#1}}
\newcommand{\BaseNTok}[1]{\textcolor[rgb]{0.00,0.00,0.81}{#1}}
\newcommand{\BuiltInTok}[1]{#1}
\newcommand{\CharTok}[1]{\textcolor[rgb]{0.31,0.60,0.02}{#1}}
\newcommand{\CommentTok}[1]{\textcolor[rgb]{0.56,0.35,0.01}{\textit{#1}}}
\newcommand{\CommentVarTok}[1]{\textcolor[rgb]{0.56,0.35,0.01}{\textbf{\textit{#1}}}}
\newcommand{\ConstantTok}[1]{\textcolor[rgb]{0.00,0.00,0.00}{#1}}
\newcommand{\ControlFlowTok}[1]{\textcolor[rgb]{0.13,0.29,0.53}{\textbf{#1}}}
\newcommand{\DataTypeTok}[1]{\textcolor[rgb]{0.13,0.29,0.53}{#1}}
\newcommand{\DecValTok}[1]{\textcolor[rgb]{0.00,0.00,0.81}{#1}}
\newcommand{\DocumentationTok}[1]{\textcolor[rgb]{0.56,0.35,0.01}{\textbf{\textit{#1}}}}
\newcommand{\ErrorTok}[1]{\textcolor[rgb]{0.64,0.00,0.00}{\textbf{#1}}}
\newcommand{\ExtensionTok}[1]{#1}
\newcommand{\FloatTok}[1]{\textcolor[rgb]{0.00,0.00,0.81}{#1}}
\newcommand{\FunctionTok}[1]{\textcolor[rgb]{0.00,0.00,0.00}{#1}}
\newcommand{\ImportTok}[1]{#1}
\newcommand{\InformationTok}[1]{\textcolor[rgb]{0.56,0.35,0.01}{\textbf{\textit{#1}}}}
\newcommand{\KeywordTok}[1]{\textcolor[rgb]{0.13,0.29,0.53}{\textbf{#1}}}
\newcommand{\NormalTok}[1]{#1}
\newcommand{\OperatorTok}[1]{\textcolor[rgb]{0.81,0.36,0.00}{\textbf{#1}}}
\newcommand{\OtherTok}[1]{\textcolor[rgb]{0.56,0.35,0.01}{#1}}
\newcommand{\PreprocessorTok}[1]{\textcolor[rgb]{0.56,0.35,0.01}{\textit{#1}}}
\newcommand{\RegionMarkerTok}[1]{#1}
\newcommand{\SpecialCharTok}[1]{\textcolor[rgb]{0.00,0.00,0.00}{#1}}
\newcommand{\SpecialStringTok}[1]{\textcolor[rgb]{0.31,0.60,0.02}{#1}}
\newcommand{\StringTok}[1]{\textcolor[rgb]{0.31,0.60,0.02}{#1}}
\newcommand{\VariableTok}[1]{\textcolor[rgb]{0.00,0.00,0.00}{#1}}
\newcommand{\VerbatimStringTok}[1]{\textcolor[rgb]{0.31,0.60,0.02}{#1}}
\newcommand{\WarningTok}[1]{\textcolor[rgb]{0.56,0.35,0.01}{\textbf{\textit{#1}}}}
\usepackage{graphicx,grffile}
\makeatletter
\def\maxwidth{\ifdim\Gin@nat@width>\linewidth\linewidth\else\Gin@nat@width\fi}
\def\maxheight{\ifdim\Gin@nat@height>\textheight\textheight\else\Gin@nat@height\fi}
\makeatother
% Scale images if necessary, so that they will not overflow the page
% margins by default, and it is still possible to overwrite the defaults
% using explicit options in \includegraphics[width, height, ...]{}
\setkeys{Gin}{width=\maxwidth,height=\maxheight,keepaspectratio}
\IfFileExists{parskip.sty}{%
\usepackage{parskip}
}{% else
\setlength{\parindent}{0pt}
\setlength{\parskip}{6pt plus 2pt minus 1pt}
}
\setlength{\emergencystretch}{3em}  % prevent overfull lines
\providecommand{\tightlist}{%
  \setlength{\itemsep}{0pt}\setlength{\parskip}{0pt}}
\setcounter{secnumdepth}{0}
% Redefines (sub)paragraphs to behave more like sections
\ifx\paragraph\undefined\else
\let\oldparagraph\paragraph
\renewcommand{\paragraph}[1]{\oldparagraph{#1}\mbox{}}
\fi
\ifx\subparagraph\undefined\else
\let\oldsubparagraph\subparagraph
\renewcommand{\subparagraph}[1]{\oldsubparagraph{#1}\mbox{}}
\fi

%%% Use protect on footnotes to avoid problems with footnotes in titles
\let\rmarkdownfootnote\footnote%
\def\footnote{\protect\rmarkdownfootnote}

%%% Change title format to be more compact
\usepackage{titling}

% Create subtitle command for use in maketitle
\providecommand{\subtitle}[1]{
  \posttitle{
    \begin{center}\large#1\end{center}
    }
}

\setlength{\droptitle}{-2em}

  \title{}
    \pretitle{\vspace{\droptitle}}
  \posttitle{}
    \author{}
    \preauthor{}\postauthor{}
    \date{}
    \predate{}\postdate{}
  

\begin{document}

\hypertarget{r-basics}{%
\section{R Basics}\label{r-basics}}

In this lesson we are going to go over the very basics of R, cover some
basic terminology, talk a little about syntax and point you to resources
for getting help.

\hypertarget{lesson-outline}{%
\subsection{Lesson Outline:}\label{lesson-outline}}

\begin{itemize}
\tightlist
\item
  \protect\hyperlink{functions-and-basic-syntax}{Functions and basic
  syntax}
\item
  \protect\hyperlink{packages}{Packages}
\item
  \protect\hyperlink{operators-and-objects}{Operators and objects}
\item
  \protect\hyperlink{getting-help}{Getting help}
\end{itemize}

\hypertarget{lesson-exercises}{%
\subsection{Lesson Exercises:}\label{lesson-exercises}}

\begin{itemize}
\tightlist
\item
  \protect\hyperlink{exercise-21}{Exercise 2.1}
\end{itemize}

\hypertarget{functions-and-basic-syntax}{%
\subsection{Functions and basic
syntax}\label{functions-and-basic-syntax}}

R is a functional programming language and as such, most everything you
do uses a function.

The basic syntax of function follows the form:
\texttt{function\_name(arg1,\ arg2,\ ...)}. With the base install, you
will gain access to many (3034 functions, to be exact). Some examples:

\begin{Shaded}
\begin{Highlighting}[]
\CommentTok{#Print}
\KeywordTok{print}\NormalTok{(}\StringTok{"hello world!"}\NormalTok{)}
\end{Highlighting}
\end{Shaded}

\begin{verbatim}
## [1] "hello world!"
\end{verbatim}

\begin{Shaded}
\begin{Highlighting}[]
\CommentTok{#A sequence}
\KeywordTok{seq}\NormalTok{(}\DecValTok{1}\NormalTok{,}\DecValTok{10}\NormalTok{)}
\end{Highlighting}
\end{Shaded}

\begin{verbatim}
##  [1]  1  2  3  4  5  6  7  8  9 10
\end{verbatim}

\begin{Shaded}
\begin{Highlighting}[]
\CommentTok{#Random normal numbers}
\KeywordTok{rnorm}\NormalTok{(}\DecValTok{100}\NormalTok{, }\DataTypeTok{mean =} \DecValTok{10}\NormalTok{, }\DataTypeTok{sd =} \DecValTok{2}\NormalTok{)}
\end{Highlighting}
\end{Shaded}

\begin{verbatim}
##   [1] 10.920410  9.513012 11.861086 14.863490  8.379840 13.141891  9.446537
##   [8]  8.022032  9.465661  8.772420 13.344523  9.118743 11.891361  9.612997
##  [15]  9.017080 10.226274  8.486031  6.841061 10.848418  8.932206  5.553944
##  [22]  8.752397  9.736154 13.455697  5.119204  9.982638  9.059941  8.795523
##  [29]  6.981665  8.044743  9.222369  9.401621  5.781320 10.480465 11.466891
##  [36]  6.971664  7.482965  8.127270  9.443446  8.659414  6.650244  8.101129
##  [43] 12.354610 12.422165 11.299381  8.624484 10.071697 11.545935  8.203265
##  [50]  9.284808  9.324787 11.729616  9.393644  9.497394 11.158317 10.074143
##  [57]  8.745845  8.455896 11.689187  9.739108  8.052527  7.124063  6.310835
##  [64] 10.877433 13.526127  9.810944  6.981618  7.402550  9.043106  8.284713
##  [71] 13.555172  9.792659 10.878405  7.276007 10.122262 12.079535  7.506201
##  [78] 11.190019 10.375745 11.711439  8.880767 11.818784 10.420440 10.057292
##  [85] 10.314723  8.253971 10.394187  8.835908  7.187634 10.845006 11.841143
##  [92]  9.217214 10.086156  6.801660 12.951945 10.548676  8.985280 11.155482
##  [99] 11.262077 13.060215
\end{verbatim}

\begin{Shaded}
\begin{Highlighting}[]
\CommentTok{#Mean}
\KeywordTok{mean}\NormalTok{(}\KeywordTok{rnorm}\NormalTok{(}\DecValTok{100}\NormalTok{))}
\end{Highlighting}
\end{Shaded}

\begin{verbatim}
## [1] 0.01657787
\end{verbatim}

\begin{Shaded}
\begin{Highlighting}[]
\CommentTok{#Sum}
\KeywordTok{sum}\NormalTok{(}\KeywordTok{rnorm}\NormalTok{(}\DecValTok{100}\NormalTok{))}
\end{Highlighting}
\end{Shaded}

\begin{verbatim}
## [1] 0.3787673
\end{verbatim}

\hypertarget{a-few-side-notes}{%
\subsubsection{A few side notes}\label{a-few-side-notes}}

There are several other characters that commonly show up in R code.
These are:

\begin{verbatim}
# - comments
() - wraps function arguments and order of operations
[] - indexing
{} - grouping code
\end{verbatim}

The \texttt{\#} indicates a comment. You can put whatever else you'd
like after this, but on the same line as the \texttt{\#}. R will not
evaluate it. Multiple \texttt{\#\#\#\#\#}, are still just seen as a
comment. When commenting your code, err on the side of too much! Also,
you will see \texttt{()}, \texttt{{[}{]}}, and \texttt{\{\}} used in R
code. The \texttt{()} indicates a function (almost always), the
\texttt{{[}{]}} indicates indexing (grabbing values by the location),
and the \texttt{\{\}} groups code that is meant to be run together and
is usually used when programming functions in R.

\hypertarget{packages}{%
\subsection{Packages}\label{packages}}

The base install of R is quite powerful, but you will soon have a need
or desire to go beyond this. Packages provide this ability. They are a
standardized method for extending R with new methods, techniques, and
programming functionality. There is a lot to say about packages
regarding finding them, using them, etc., but for now let's focus just
on the basics.

\hypertarget{cran}{%
\subsubsection{CRAN}\label{cran}}

One of the reasons for R's popularity is CRAN,
\href{http://cran.r-project.org/}{The Comprehensive R Archive Network}.
This is where you download R and also where most will gain access to
packages (there are other places, but that is for later). Not much else
to say about this now other than to be aware of it. As of 2019-11-26,
there are 15242 packages on CRAN!

\hypertarget{installing-packages}{%
\subsubsection{Installing packages}\label{installing-packages}}

When a package gets installed, that means the source is downloaded and
put into your library. A default library location is set for you so no
need to worry about that. In fact on Windows most of this is pretty
automatic. Let's give it a shot.

\begin{Shaded}
\begin{Highlighting}[]
\CommentTok{#Installing Packages from CRAN}
\CommentTok{#Install dplyr and ggplot2}
\KeywordTok{install.packages}\NormalTok{(}\StringTok{"ggplot2"}\NormalTok{)}
\KeywordTok{install.packages}\NormalTok{(}\StringTok{"dplyr"}\NormalTok{)}

\CommentTok{#You can also put more than one in like}
\KeywordTok{install.packages}\NormalTok{(}\KeywordTok{c}\NormalTok{(}\StringTok{"quickmapr"}\NormalTok{, }\StringTok{"formatR"}\NormalTok{))}
\end{Highlighting}
\end{Shaded}

Now a couple of words of warning for library locations on EPA windows
machines. If R was installed correctly, your library should reside in
something like \texttt{C:/Program\ Files/R/R-3.6.1/library}. If you type
\texttt{.libPaths()} and get a vastly different result
(e.g.~\texttt{Blah\textbackslash{}Blah\textbackslash{}Net\ MyDocuments\textbackslash{}Blah}),
then you are going to have some problems. We can fix this by adding a
file, \texttt{Renviron.site} to
\texttt{C:\textbackslash{}Program\ Files\textbackslash{}R\textbackslash{}R-3.6.1\textbackslash{}etc}.
That file should have the following lines in it

\begin{verbatim}
HOME=C:/Program Files/R/R-3.6.1
R_USER=C:/Program Files/R/R-3.6.1
\end{verbatim}

On your next start up of R, the library path should look better!

\hypertarget{using-packages}{%
\subsubsection{Using packages}\label{using-packages}}

One source of confusion that many have is when they cannot access a
package that they just installed. This is because getting to this point
requires an extra step, loading (or attaching) the package.

\begin{Shaded}
\begin{Highlighting}[]
\CommentTok{#Loading packages into your library}
\CommentTok{#Add libraries to your R Session}
\KeywordTok{library}\NormalTok{(}\StringTok{"ggplot2"}\NormalTok{)}
\KeywordTok{library}\NormalTok{(}\StringTok{"dplyr"}\NormalTok{)}

\CommentTok{#You can also access functions without loading by using package::function}
\NormalTok{dplyr}\OperatorTok{::}\NormalTok{mutate}
\end{Highlighting}
\end{Shaded}

\begin{verbatim}
## function (.data, ...) 
## {
##     UseMethod("mutate")
## }
## <bytecode: 0x00000233bbc3ea98>
## <environment: namespace:dplyr>
\end{verbatim}

You will often see people use \texttt{require()} to load a package. It
is better form to not do this. For a more detailed explanation of why
\texttt{library()} and not \texttt{require()} see
\href{http://yihui.name/en/2014/07/library-vs-require/.}{Yihui Xie's
post on the subject}

And now for a little pedantry. You will often hear people use the terms
``library'' and ``package'' interchangeably. This is not correct. A
package is what is submitted to CRAN, it is what contains a group of
functions that address a common problem, and it is what has allowed R to
expand. A library is, more or less, where your packages are stored. You
have a path to that library and this is where R puts new packages that
you install (e.g.~via \texttt{install.packages()}). These two terms are
related, but most certainly different. Apologies up front if I slip and
use one when I actually mean the other\ldots{}

\hypertarget{operators-and-objects}{%
\subsection{Operators and objects}\label{operators-and-objects}}

As I mentioned above, the console and using R interactively is very
powerful. We will do this quite a bit. Let's spend a little time playing
around in the console and learn a few new functions.

R can be used as a calculator and a way to compare values. Some examples
of the basic operators:

\begin{Shaded}
\begin{Highlighting}[]
\CommentTok{#A really powerful calculator!}
\DecValTok{1} \OperatorTok{+}\StringTok{ }\DecValTok{1} \CommentTok{#Add}
\end{Highlighting}
\end{Shaded}

\begin{verbatim}
## [1] 2
\end{verbatim}

\begin{Shaded}
\begin{Highlighting}[]
\DecValTok{10} \OperatorTok{-}\StringTok{ }\DecValTok{4} \CommentTok{#Subtract}
\end{Highlighting}
\end{Shaded}

\begin{verbatim}
## [1] 6
\end{verbatim}

\begin{Shaded}
\begin{Highlighting}[]
\DecValTok{3} \OperatorTok{*}\StringTok{ }\DecValTok{2} \CommentTok{#Multiply}
\end{Highlighting}
\end{Shaded}

\begin{verbatim}
## [1] 6
\end{verbatim}

\begin{Shaded}
\begin{Highlighting}[]
\DecValTok{3} \OperatorTok{^}\StringTok{ }\DecValTok{3} \CommentTok{#Exponents}
\end{Highlighting}
\end{Shaded}

\begin{verbatim}
## [1] 27
\end{verbatim}

\begin{Shaded}
\begin{Highlighting}[]
\DecValTok{100} \OperatorTok{/}\StringTok{ }\DecValTok{10} \CommentTok{#Divide}
\end{Highlighting}
\end{Shaded}

\begin{verbatim}
## [1] 10
\end{verbatim}

\begin{Shaded}
\begin{Highlighting}[]
\DecValTok{5} \OperatorTok\StringTok{ }\DecValTok{2} \CommentTok{#Modulus}
\end{Highlighting}
\end{Shaded}

\begin{verbatim}
## [1] 1
\end{verbatim}

\begin{Shaded}
\begin{Highlighting}[]
\DecValTok{5} \OperatorTok{>}\StringTok{ }\DecValTok{2} \CommentTok{#Greater than}
\end{Highlighting}
\end{Shaded}

\begin{verbatim}
## [1] TRUE
\end{verbatim}

\begin{Shaded}
\begin{Highlighting}[]
\DecValTok{4} \OperatorTok{<}\StringTok{ }\DecValTok{5} \CommentTok{#Less than}
\end{Highlighting}
\end{Shaded}

\begin{verbatim}
## [1] TRUE
\end{verbatim}

\begin{Shaded}
\begin{Highlighting}[]
\DecValTok{5} \OperatorTok{<=}\StringTok{ }\DecValTok{5} \CommentTok{#Less than or equal}
\end{Highlighting}
\end{Shaded}

\begin{verbatim}
## [1] TRUE
\end{verbatim}

\begin{Shaded}
\begin{Highlighting}[]
\DecValTok{8} \OperatorTok{>=}\StringTok{ }\DecValTok{2} \CommentTok{#Greater than or equal}
\end{Highlighting}
\end{Shaded}

\begin{verbatim}
## [1] TRUE
\end{verbatim}

\begin{Shaded}
\begin{Highlighting}[]
\DecValTok{2} \OperatorTok{==}\StringTok{ }\DecValTok{2} \CommentTok{#Equality: notice that it is TWO equal signs!}
\end{Highlighting}
\end{Shaded}

\begin{verbatim}
## [1] TRUE
\end{verbatim}

\begin{Shaded}
\begin{Highlighting}[]
\DecValTok{5} \OperatorTok{!=}\StringTok{ }\DecValTok{7} \CommentTok{#Not Equals}
\end{Highlighting}
\end{Shaded}

\begin{verbatim}
## [1] TRUE
\end{verbatim}

That's neat, but so what\ldots{}

Well, it could be interesting to do something with those values and save
them for re-use. We can do that with objects (everything in R is an
object) and use the assignment operator, \texttt{\textless{}-}. Know
that object names cannot start with a number, contain spaces, or (most)
special characters. Underscores and periods are allowed.

\textbf{NOTE:} If you have experience with other, object-oriented
languages, then just be aware that R objects, at least the general use
of the term, are different.

\begin{Shaded}
\begin{Highlighting}[]
\CommentTok{#Numeric assignment}
\NormalTok{x <-}\StringTok{ }\DecValTok{5}
\NormalTok{x}
\end{Highlighting}
\end{Shaded}

\begin{verbatim}
## [1] 5
\end{verbatim}

\begin{Shaded}
\begin{Highlighting}[]
\NormalTok{y <-}\StringTok{ }\NormalTok{x }\OperatorTok{+}\StringTok{ }\DecValTok{1}
\NormalTok{y}
\end{Highlighting}
\end{Shaded}

\begin{verbatim}
## [1] 6
\end{verbatim}

\begin{Shaded}
\begin{Highlighting}[]
\NormalTok{z <-}\StringTok{ }\NormalTok{x }\OperatorTok{+}\StringTok{ }\NormalTok{y}
\NormalTok{z}
\end{Highlighting}
\end{Shaded}

\begin{verbatim}
## [1] 11
\end{verbatim}

\begin{Shaded}
\begin{Highlighting}[]
\CommentTok{#Character}
\NormalTok{a <-}\StringTok{ "Bob"}
\NormalTok{a}
\end{Highlighting}
\end{Shaded}

\begin{verbatim}
## [1] "Bob"
\end{verbatim}

\begin{Shaded}
\begin{Highlighting}[]
\NormalTok{b <-}\StringTok{ "Sue"}
\NormalTok{b}
\end{Highlighting}
\end{Shaded}

\begin{verbatim}
## [1] "Sue"
\end{verbatim}

\begin{Shaded}
\begin{Highlighting}[]
\NormalTok{a2 <-}\StringTok{ "Larry"}
\NormalTok{a2}
\end{Highlighting}
\end{Shaded}

\begin{verbatim}
## [1] "Larry"
\end{verbatim}

Now that we have a little experience working in the console and creating
objects with \texttt{\textless{}-}, we might want to be able to do some
additional things to navigate around, look at these objects etc. You can
do a lot of this directly in RStudio in the Environment, History pane
which is likely in the upper right corner of the window. Alternatively,
you can explore your current environment via the console. Some functions
that you might find useful for working with your R workspace:

\begin{Shaded}
\begin{Highlighting}[]
\CommentTok{#List all objects in current workspace}
\KeywordTok{ls}\NormalTok{() }
\KeywordTok{ls}\NormalTok{(}\DataTypeTok{pattern =} \StringTok{"a"}\NormalTok{)}

\CommentTok{#Remove an object}
\KeywordTok{rm}\NormalTok{(x)}

\CommentTok{#Save your workspace}
\CommentTok{#Saves the whole thing to a file called lesson2.RData}
\KeywordTok{save.image}\NormalTok{(}\StringTok{"lesson2.RData"}\NormalTok{) }
\CommentTok{#Saves just the a and y objects to a file called lesson2_ay.RData}
\KeywordTok{save}\NormalTok{(a, y, }\DataTypeTok{file =} \StringTok{"lesson2_ay.RData"}\NormalTok{)}
\end{Highlighting}
\end{Shaded}

This is probably a good spot to bring up quotes vs no quotes around
arguments in a function. This is a very common stumbling block. The
general rule is that no quotes are used only when referring to an object
that currently exists. Quotes are used in all other cases. For instance
in \texttt{save(a,\ y,\ file\ =\ "lesson2\_ay.RData")} the objects
\texttt{a} and \texttt{y} are not quoted because they are objects in the
workspace. \texttt{file} is an argument of save and argument names are
never quoted. We quote the name of the file ``lesson2\_ay.RData''
because it is not an R object but the name of a file to be created. You
will likely still have some issues with this. My recommendation is to
think about if it is an object in your R workspace or not. If so, no
quotes! This isn't foolproof, but works well most of the time.

Next thing you might want to do is navigate around your files and
directories. While you can do this directly from the console, it is
going to be better practice to mostly use RStudio projects to manage
your folders, working directory etc. You can also navigate using the
Files, etc. pane. \#\#\# A quick word about assignment We have now seen
how to assign object in R using the assignment operator,
\texttt{\textless{}-}. Assignment can also be done with \texttt{=}. I
suggest that you don't use this, not for ``correct'' syntax reasons, but
for readability and code style reasons. Use \texttt{\textless{}-} for
assignment (fun note, \texttt{-\textgreater{}} also works but the object
it is assigned to lives on the right side of the operator). Reserve
\texttt{=} for assigning arguments inside of functions
(e.g.~\texttt{rnorm(10,\ mean\ =\ 5)}).

\hypertarget{getting-help}{%
\subsection{Getting help}\label{getting-help}}

Being able to find help and interpret that help is probably one of the
most important skills for learning a new language. R is no different.
Help on functions and packages can be accessed directly from R, can be
found on CRAN and other official R resources, searched on Google, found
on StackOverflow, or from any number of fantastic online resources. I
will cover a few of these here.

\hypertarget{help-from-the-console}{%
\subsubsection{Help from the console}\label{help-from-the-console}}

Getting help from the console is straightforward and can be done
numerous ways.

\begin{Shaded}
\begin{Highlighting}[]
\CommentTok{#Using the help command/shortcut}
\CommentTok{#When you know the name of a function}
\KeywordTok{help}\NormalTok{(}\StringTok{"print"}\NormalTok{) }\CommentTok{#Help on the print command}
\NormalTok{?print }\CommentTok{#Help on the print command using the `?` shortcut}

\CommentTok{#When you know the name of the package}
\KeywordTok{help}\NormalTok{(}\DataTypeTok{package=}\StringTok{"dplyr"}\NormalTok{) }\CommentTok{#Help on the package `dplyr`}

\CommentTok{#Don't know the exact name or just part of it}
\KeywordTok{apropos}\NormalTok{(}\StringTok{"print"}\NormalTok{) }\CommentTok{#Returns all available functions with "print" in the name}
\NormalTok{??print }\CommentTok{#Shortcut, but also searches demos and vignettes in a formatted page}
\end{Highlighting}
\end{Shaded}

\hypertarget{official-r-resources}{%
\subsubsection{Official R Resources}\label{official-r-resources}}

In addition to help from within R itself, CRAN and the R-Project have
many resources available for support. Two of the most notable are the
mailing lists and the \href{http://cran.r-project.org/web/views/}{task
views}.

\begin{itemize}
\tightlist
\item
  \href{https://stat.ethz.ch/mailman/listinfo/r-help}{R Help Mailing
  List}: The main mailing list for R help. Can be a bit daunting and
  some (although not most) senior folks can be, um, curmudgeonly\ldots{}
\item
  \href{https://stat.ethz.ch/mailman/listinfo/r-sig-ecology}{R-sig-ecology}:
  A special interest group for use of R in ecology. Less daunting the
  the main help with participation from some big names in ecological
  modelling and statistics (e.g., Ben Bolker, Gavin Simpson, and Phil
  Dixon). One of the moderators is great, the other is a bit of a jerk
  (it's me).
\item
  \href{http://cran.r-project.org/web/views/Environmetrics.html}{Environmetrics
  Task View}: Task views are great in that they provide an annotated
  list of packages relevant to a particular field. This one is
  maintained by Gavin Simpson and has great info on packages relevant to
  much of the work at EPA.
\item
  \href{http://cran.r-project.org/web/views/Spatial.html}{Spatial
  Analysis Task View}: One I use a lot that lists all the relevant
  packages for spatial analysis, GIS, and Remote Sensing in R.
\end{itemize}

\hypertarget{google-and-stackoverflow}{%
\subsubsection{Google and
StackOverflow}\label{google-and-stackoverflow}}

While the resources already mentioned are useful, often the quickest way
is to just turn to Google. However, a search for ``R'' is a bit
challenging. A few ways around this. Google works great if you search
for a given package or function name. You can search for mailing lists
directly (i.e.~``R-sig-geo''). An R specific search tool,
\href{http://www.rseek.org/}{RSeek.org}, has been created to facilitate
this.

One specific resource that I use quite a bit is
\href{http://stackoverflow.com/questions/tagged/r}{StackOverflow with
the `r' tag}. StackOverflow is a discussion forum for all things related
to programming. You can then use this tag and the search functions in
StackOverflow and find answers to almost anything you can think of.

\hypertarget{other-resources}{%
\subsubsection{Other Resources}\label{other-resources}}

As I mention earlier, there are TOO many resources to mention and
everyone has their favorites. Below are just a few that I like.

\begin{itemize}
\tightlist
\item
  \href{http://r4ds.had.co.nz/}{R For Data Science}: Another book by
  Hadley and Garrett Grolemund. First reference I suggest for new (and
  experienced) R Learners, especially for all things Tidy.
\item
  \href{http://adv-r.had.co.nz/}{Advanced R}: Web home of Hadley
  Wickham's new book. Gets into more advanced topics, but also covers
  the basics in a great way.
\item
  \href{http://rforcats.net/}{R For Cats}: Basic introduction site,
  meant to be a gentle and light-hearted introduction.
\item
  \href{http://cran.r-project.org/doc/contrib/Short-refcard.pdf}{CRAN
  Cheatsheets}: A good cheat sheet from the official source
\item
  \href{http://www.rstudio.com/resources/cheatsheets/}{RStudio
  Cheatsheets}: Additional cheat sheets from RStudio. I am especially
  fond of the data wrangling one.
\end{itemize}

\hypertarget{exercise-2.1}{%
\subsection{Exercise 2.1}\label{exercise-2.1}}

We should still have our \texttt{nla\_analysis.R} file open. We will be
working with this as we go through the rest of the excercises.

Take a look at this file and with the person sitting next to you find
the following:

\begin{enumerate}
\def\labelenumi{\arabic{enumi}.}
\tightlist
\item
  Find the \texttt{read\_csv()} function. What lines is it on? What is
  the argument?
\item
  Now find the lines on which you think we install packages and load
  libraries. There is the fancy way (lines 19-24) and a straight up way
  (lines 29-32). Talk through in your own words what each of these is
  doing
\item
  Add a line of code after line 24 to install the package
  \texttt{lubridate}. Add a line after line 33 to load
  \texttt{lubridate}.
\item
  Bring up the package level help for the \texttt{lubridate} package.
  What does this package do?
\end{enumerate}


\end{document}
