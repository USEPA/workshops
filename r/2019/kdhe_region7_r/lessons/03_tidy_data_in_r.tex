\documentclass[]{article}
\usepackage{lmodern}
\usepackage{amssymb,amsmath}
\usepackage{ifxetex,ifluatex}
\usepackage{fixltx2e} % provides \textsubscript
\ifnum 0\ifxetex 1\fi\ifluatex 1\fi=0 % if pdftex
  \usepackage[T1]{fontenc}
  \usepackage[utf8]{inputenc}
\else % if luatex or xelatex
  \ifxetex
    \usepackage{mathspec}
  \else
    \usepackage{fontspec}
  \fi
  \defaultfontfeatures{Ligatures=TeX,Scale=MatchLowercase}
\fi
% use upquote if available, for straight quotes in verbatim environments
\IfFileExists{upquote.sty}{\usepackage{upquote}}{}
% use microtype if available
\IfFileExists{microtype.sty}{%
\usepackage{microtype}
\UseMicrotypeSet[protrusion]{basicmath} % disable protrusion for tt fonts
}{}
\usepackage[margin=1in]{geometry}
\usepackage{hyperref}
\hypersetup{unicode=true,
            pdfborder={0 0 0},
            breaklinks=true}
\urlstyle{same}  % don't use monospace font for urls
\usepackage{color}
\usepackage{fancyvrb}
\newcommand{\VerbBar}{|}
\newcommand{\VERB}{\Verb[commandchars=\\\{\}]}
\DefineVerbatimEnvironment{Highlighting}{Verbatim}{commandchars=\\\{\}}
% Add ',fontsize=\small' for more characters per line
\usepackage{framed}
\definecolor{shadecolor}{RGB}{248,248,248}
\newenvironment{Shaded}{\begin{snugshade}}{\end{snugshade}}
\newcommand{\AlertTok}[1]{\textcolor[rgb]{0.94,0.16,0.16}{#1}}
\newcommand{\AnnotationTok}[1]{\textcolor[rgb]{0.56,0.35,0.01}{\textbf{\textit{#1}}}}
\newcommand{\AttributeTok}[1]{\textcolor[rgb]{0.77,0.63,0.00}{#1}}
\newcommand{\BaseNTok}[1]{\textcolor[rgb]{0.00,0.00,0.81}{#1}}
\newcommand{\BuiltInTok}[1]{#1}
\newcommand{\CharTok}[1]{\textcolor[rgb]{0.31,0.60,0.02}{#1}}
\newcommand{\CommentTok}[1]{\textcolor[rgb]{0.56,0.35,0.01}{\textit{#1}}}
\newcommand{\CommentVarTok}[1]{\textcolor[rgb]{0.56,0.35,0.01}{\textbf{\textit{#1}}}}
\newcommand{\ConstantTok}[1]{\textcolor[rgb]{0.00,0.00,0.00}{#1}}
\newcommand{\ControlFlowTok}[1]{\textcolor[rgb]{0.13,0.29,0.53}{\textbf{#1}}}
\newcommand{\DataTypeTok}[1]{\textcolor[rgb]{0.13,0.29,0.53}{#1}}
\newcommand{\DecValTok}[1]{\textcolor[rgb]{0.00,0.00,0.81}{#1}}
\newcommand{\DocumentationTok}[1]{\textcolor[rgb]{0.56,0.35,0.01}{\textbf{\textit{#1}}}}
\newcommand{\ErrorTok}[1]{\textcolor[rgb]{0.64,0.00,0.00}{\textbf{#1}}}
\newcommand{\ExtensionTok}[1]{#1}
\newcommand{\FloatTok}[1]{\textcolor[rgb]{0.00,0.00,0.81}{#1}}
\newcommand{\FunctionTok}[1]{\textcolor[rgb]{0.00,0.00,0.00}{#1}}
\newcommand{\ImportTok}[1]{#1}
\newcommand{\InformationTok}[1]{\textcolor[rgb]{0.56,0.35,0.01}{\textbf{\textit{#1}}}}
\newcommand{\KeywordTok}[1]{\textcolor[rgb]{0.13,0.29,0.53}{\textbf{#1}}}
\newcommand{\NormalTok}[1]{#1}
\newcommand{\OperatorTok}[1]{\textcolor[rgb]{0.81,0.36,0.00}{\textbf{#1}}}
\newcommand{\OtherTok}[1]{\textcolor[rgb]{0.56,0.35,0.01}{#1}}
\newcommand{\PreprocessorTok}[1]{\textcolor[rgb]{0.56,0.35,0.01}{\textit{#1}}}
\newcommand{\RegionMarkerTok}[1]{#1}
\newcommand{\SpecialCharTok}[1]{\textcolor[rgb]{0.00,0.00,0.00}{#1}}
\newcommand{\SpecialStringTok}[1]{\textcolor[rgb]{0.31,0.60,0.02}{#1}}
\newcommand{\StringTok}[1]{\textcolor[rgb]{0.31,0.60,0.02}{#1}}
\newcommand{\VariableTok}[1]{\textcolor[rgb]{0.00,0.00,0.00}{#1}}
\newcommand{\VerbatimStringTok}[1]{\textcolor[rgb]{0.31,0.60,0.02}{#1}}
\newcommand{\WarningTok}[1]{\textcolor[rgb]{0.56,0.35,0.01}{\textbf{\textit{#1}}}}
\usepackage{graphicx,grffile}
\makeatletter
\def\maxwidth{\ifdim\Gin@nat@width>\linewidth\linewidth\else\Gin@nat@width\fi}
\def\maxheight{\ifdim\Gin@nat@height>\textheight\textheight\else\Gin@nat@height\fi}
\makeatother
% Scale images if necessary, so that they will not overflow the page
% margins by default, and it is still possible to overwrite the defaults
% using explicit options in \includegraphics[width, height, ...]{}
\setkeys{Gin}{width=\maxwidth,height=\maxheight,keepaspectratio}
\IfFileExists{parskip.sty}{%
\usepackage{parskip}
}{% else
\setlength{\parindent}{0pt}
\setlength{\parskip}{6pt plus 2pt minus 1pt}
}
\setlength{\emergencystretch}{3em}  % prevent overfull lines
\providecommand{\tightlist}{%
  \setlength{\itemsep}{0pt}\setlength{\parskip}{0pt}}
\setcounter{secnumdepth}{0}
% Redefines (sub)paragraphs to behave more like sections
\ifx\paragraph\undefined\else
\let\oldparagraph\paragraph
\renewcommand{\paragraph}[1]{\oldparagraph{#1}\mbox{}}
\fi
\ifx\subparagraph\undefined\else
\let\oldsubparagraph\subparagraph
\renewcommand{\subparagraph}[1]{\oldsubparagraph{#1}\mbox{}}
\fi

%%% Use protect on footnotes to avoid problems with footnotes in titles
\let\rmarkdownfootnote\footnote%
\def\footnote{\protect\rmarkdownfootnote}

%%% Change title format to be more compact
\usepackage{titling}

% Create subtitle command for use in maketitle
\providecommand{\subtitle}[1]{
  \posttitle{
    \begin{center}\large#1\end{center}
    }
}

\setlength{\droptitle}{-2em}

  \title{}
    \pretitle{\vspace{\droptitle}}
  \posttitle{}
    \author{}
    \preauthor{}\postauthor{}
    \date{}
    \predate{}\postdate{}
  

\begin{document}

\hypertarget{tidy-data-in-r}{%
\section{Tidy Data in R}\label{tidy-data-in-r}}

In this lesson we will cover the basics of data in R and will do so from
a somewhat opinionated viewpoint of ``Tidy Data''. There are other
paradigms and other ways to work with data in R, but focusing on Tidy
Data concepts and tools (a.k.a., The Tidyverse) gets people to a
productive place the quickest. For more on data analysis using the
Tidyverse, the best resource I know of is \href{http://r4ds.had.co.nz}{R
for Data Science}. The approaches we will cover are very much inspired
by this book.

\hypertarget{lesson-outline}{%
\subsection{Lesson Outline}\label{lesson-outline}}

\begin{itemize}
\tightlist
\item
  \protect\hyperlink{data-in-r-the-data-frame}{Data in R: The data
  frame}
\item
  \protect\hyperlink{reading-in-data}{Reading in data}
\item
  \protect\hyperlink{tidy-data}{Tidy data}
\end{itemize}

\hypertarget{exercises}{%
\subsection{Exercises}\label{exercises}}

\begin{itemize}
\tightlist
\item
  \protect\hyperlink{exercise-31}{Exercise 3.1}
\item
  \protect\hyperlink{exercise-32}{Exercise 3.2}
\item
  \protect\hyperlink{exercise-33}{Exercise 3.3}
\end{itemize}

\hypertarget{data-in-r-the-data-frame}{%
\subsection{Data in R: The data frame}\label{data-in-r-the-data-frame}}

Simply put, a data structure is a way for programming languages to
handle storing information. Like most languages, R has several
structures (vectors, matrix, lists, etc.). But R was originally built
for data analysis, so the data frame, a spreadsheet like structure with
rows and columns, is the most widely used and useful to learn first. In
addition, the data frame (or is it data.frame) is the basis for many
modern R pacakges (e.g.~the tidyverse) and getting used to it will allow
you to quickly build your R skills.

\emph{Note:} It is useful to know more about the different data
structures such as vectors, lists, and factors (a weird one that is for
catergorical data). But that is beyond what we have time for. The best
source on this information, I think, is Hadley Wickham's
\href{http://adv-r.had.co.nz/Data-structures.html}{Data Structures
Chapter in Advanced R}.

\emph{Another note:} Data types (e.g.~numeric, character, logcial, etc.)
are important to know about too, but details are more than we have time
for. Take a look at the chapter on vectors in R for Data Science, in
particular
\href{https://r4ds.had.co.nz/vectors.html\#important-types-of-atomic-vector}{Section
20.3}.

\emph{And, yet another note:} Computers aren't very good at math. Or at
least they don't deal with floating point data they way many would
think. First, any value that is not an integer, is considered a
``Double.'' These are approximations, so if we are looking to compare to
doubles, we might not always get the result we are expecting. Again,
R4DS is a great read on this:
\href{https://r4ds.had.co.nz/vectors.html\#numeric}{Section on Numeric
Vectors}. But also see
\href{https://blog.revolutionanalytics.com/2009/11/floatingpoint-errors-explained.html}{this
take from Revolution Analytics} and
\href{https://www.techradar.com/news/computing/why-computers-suck-at-maths-644771/2}{techradar}.

\emph{Last note, I promise:} Your elementary education was wrong. In
other words rounding 2.5 is 2 and not 3, but rounding 3.5 is 4. There
are actually good reasons for this. Read up on the IEEE 754 standard
\href{https://en.wikipedia.org/wiki/IEEE_754\#Rounding_rules}{rules on
rounding}.

\hypertarget{build-a-data-frame}{%
\subsubsection{Build a data frame}\label{build-a-data-frame}}

Best way to learn what a data frame is is to look at one. Let's now
build a simple data frame from scratch with the \texttt{data.frame()}
function. This is mostly a teaching excercise as we will use the
function very little in the excercises to come.

\begin{Shaded}
\begin{Highlighting}[]
\CommentTok{# Our first data frame}

\NormalTok{my_df <-}\StringTok{ }\KeywordTok{data.frame}\NormalTok{(}\DataTypeTok{names =} \KeywordTok{c}\NormalTok{(}\StringTok{"joe"}\NormalTok{,}\StringTok{"jenny"}\NormalTok{,}\StringTok{"bob"}\NormalTok{,}\StringTok{"sue"}\NormalTok{), }
                    \DataTypeTok{age =} \KeywordTok{c}\NormalTok{(}\DecValTok{45}\NormalTok{, }\DecValTok{27}\NormalTok{, }\DecValTok{38}\NormalTok{,}\DecValTok{51}\NormalTok{), }
                    \DataTypeTok{knows_r =} \KeywordTok{c}\NormalTok{(}\OtherTok{FALSE}\NormalTok{, }\OtherTok{TRUE}\NormalTok{, }\OtherTok{TRUE}\NormalTok{,}\OtherTok{FALSE}\NormalTok{), }
                    \DataTypeTok{stringsAsFactors =} \OtherTok{FALSE}\NormalTok{)}
\NormalTok{my_df}
\end{Highlighting}
\end{Shaded}

\begin{verbatim}
##   names age knows_r
## 1   joe  45   FALSE
## 2 jenny  27    TRUE
## 3   bob  38    TRUE
## 4   sue  51   FALSE
\end{verbatim}

That created a data frame with 3 columns (names, age, knows\_r) and four
rows. For each row we have some information on the name of an individual
(stored as a character/string), their age (stored as a numeric value),
and a column indicating if they know R or not (stored as a
boolean/logical).

If you've worked with data before in a spreadsheet or from a table in a
database, this rectangular structure should look somewhat familiar. One
way (there are many!) we can access the different parts of the data
frame is like:

\begin{Shaded}
\begin{Highlighting}[]
\CommentTok{# Use the dollar sign to get a column}
\NormalTok{my_df}\OperatorTok{$}\NormalTok{age}
\end{Highlighting}
\end{Shaded}

\begin{verbatim}
## [1] 45 27 38 51
\end{verbatim}

\begin{Shaded}
\begin{Highlighting}[]
\CommentTok{# Grab a row with indexing}
\NormalTok{my_df[}\DecValTok{2}\NormalTok{,]}
\end{Highlighting}
\end{Shaded}

\begin{verbatim}
##   names age knows_r
## 2 jenny  27    TRUE
\end{verbatim}

At this point, we have:

\begin{itemize}
\tightlist
\item
  built a data frame from scratch
\item
  seen rows and columns
\item
  heard about ``rectangular'' structure
\item
  seen how to get a row and a column
\end{itemize}

The purpose of all this was to introduce the concept of the data frame.
Moving forward we will use other tools to read in data, but the end
result will be the same: a data frame with rows (i.e.~observations) and
columns (i.e.~variables).

\hypertarget{reading-in-data}{%
\subsection{Reading in data}\label{reading-in-data}}

Completely creating a data frame from scratch is useful (especially when
you start writing your own functions), but more often than not data is
stored in an external file that you need to read into R. These may be
delimited text files, spreadsheets, relational databases, SAS files
\ldots{} You get the idea. Instead of treating this subject
exhaustively, we will focus just on a single file type, the
\texttt{.csv} file, that is very commonly encountered and (usually) easy
to create from other file types. For this, we will use the Tidyverse way
to do this and use \texttt{read\_csv()} from the \texttt{readr} pacakge.

The \texttt{read\_csv()} function is a re-imagined version of the base R
fucntion, \texttt{read.csv()}. This command assumes a header row with
column names and that the delimiter is a comma. The expected no data
value is NA and by default, strings are NOT converted to factors. This
is a big benefit to using \texttt{read\_csv()} as opposed to
\texttt{read.csv()}. Additionally, \texttt{read\_csv()} has some
performance enhancements that make it preferrable when working with
larger data sets. In my limited experience it is about 45\% faster than
the base R options. For instance a \textasciitilde200 MB file with
hundreds of columns and a couple hundred thousand rows took
\textasciitilde14 seconds to read in with \texttt{read\_csv()} and about
24 seconds with \texttt{read.csv()}. As a comparison at 45 seconds Excel
had only opened 25\% of the file!

Source files for \texttt{read\_csv()} can either be on a local hard
drive or, and this is pretty cool, on the web. We will be using the
former for our examples and exercises. If you had a file available from
a URL it would be accessed like
\texttt{mydf\ \textless{}-\ read.csv("https://example.com/my\_cool\_file.csv")}.
As an aside, paths and the use of forward vs back slash is important. R
is looking for forward slashes (``/''), or unix-like paths. You can use
these in place of the back slash and be fine. You can use a back slash
but it needs to be a double back slash
(``\textbackslash\textbackslash{}''). This is becuase the single
backslash in an escape character that is used to indicate things like
newlines or tabs.

For today's workshop we will focus on both grabbing data from a local
file and from a URL, we already have an example of this in our
\texttt{nla\_analysis.R}. In that file look for the line where we use
\texttt{read\_csv()}

For your convenience, it looks like:

\begin{Shaded}
\begin{Highlighting}[]
\NormalTok{nla_wq_all <-}\StringTok{ }\KeywordTok{read_csv}\NormalTok{(}\StringTok{"nla2007_chemical_conditionestimates_20091123.csv"}\NormalTok{)}
\end{Highlighting}
\end{Shaded}

And now we can take a look at our data frame

\begin{Shaded}
\begin{Highlighting}[]
\NormalTok{nla_wq_all}
\end{Highlighting}
\end{Shaded}

\begin{verbatim}
## # A tibble: 1,252 x 51
##    SITE_ID VISIT_NO SITE_TYPE LAKE_SAMP TNT   LAT_DD LON_DD ST    EPA_REG
##    <chr>      <dbl> <chr>     <chr>     <chr>  <dbl>  <dbl> <chr> <chr>  
##  1 NLA066~        1 PROB_Lake Target_S~ Targ~   49.0 -114.  MT    Region~
##  2 NLA066~        1 PROB_Lake Target_S~ Targ~   33.0  -80.0 SC    Region~
##  3 NLA066~        2 PROB_Lake Target_S~ Targ~   33.0  -80.0 SC    Region~
##  4 NLA066~        1 PROB_Lake Target_S~ Targ~   28.0  -97.9 TX    Region~
##  5 NLA066~        2 PROB_Lake Target_S~ Targ~   28.0  -97.9 TX    Region~
##  6 NLA066~        1 PROB_Lake Target_S~ Targ~   37.4 -108.  CO    Region~
##  7 NLA066~        2 PROB_Lake Target_S~ Targ~   37.4 -108.  CO    Region~
##  8 NLA066~        1 PROB_Lake Target_S~ Targ~   43.9 -115.  ID    Region~
##  9 NLA066~        2 PROB_Lake Target_S~ Targ~   43.9 -115.  ID    Region~
## 10 NLA066~        1 PROB_Lake Target_S~ Targ~   41.7  -73.1 CT    Region~
## # ... with 1,242 more rows, and 42 more variables: AREA_CAT7 <chr>,
## #   NESLAKE <chr>, STRATUM <chr>, PANEL <chr>, DSGN_CAT <chr>,
## #   MDCATY <dbl>, WGT <dbl>, WGT_NLA <dbl>, ADJWGT_CAT <chr>, URBAN <chr>,
## #   WSA_ECO3 <chr>, WSA_ECO9 <chr>, ECO_LEV_3 <dbl>, NUT_REG <chr>,
## #   NUTREG_NAME <chr>, ECO_NUTA <chr>, LAKE_ORIGIN <chr>,
## #   ECO3_X_ORIGIN <chr>, REF_CLUSTER <chr>, RT_NLA <chr>, HUC_2 <dbl>,
## #   HUC_8 <dbl>, FLAG_INFO <chr>, COMMENT_INFO <chr>, SAMPLED <chr>,
## #   SAMPLED_CHEM <chr>, INDXSAMP_CHEM <chr>, PTL <dbl>, NTL <dbl>,
## #   TURB <dbl>, ANC <dbl>, DOC <dbl>, COND <dbl>, SAMPLED_CHLA <chr>,
## #   INDXSAMP_CHLA <chr>, CHLA <dbl>, PTL_COND <chr>, NTL_COND <chr>,
## #   CHLA_COND <chr>, TURB_COND <chr>, ANC_COND <chr>, SALINITY_COND <chr>
\end{verbatim}

\hypertarget{other-ways-to-read-in-data}{%
\subsubsection{Other ways to read in
data}\label{other-ways-to-read-in-data}}

There are many ways to read in data with R. If you have questions about
this, please let Jeff know. He's happy to chat more about it. Before we
move on though, I will show an example of one other way we can do this.
Since Excel spreadsheets are so ubiquitous we need a reliable way to
read in data stored in an excel spreadsheet. There are a variety of
packages that provide this capability, but by far the best (IMHO) is
\texttt{readxl} which is part of the Tidyverse. This is how we read in
an File:

\begin{Shaded}
\begin{Highlighting}[]
\CommentTok{# You'll very likely need to install it first!!!  How'd we do that?}
\KeywordTok{library}\NormalTok{(readxl)}
\NormalTok{nla_wq_excel <-}\StringTok{ }\KeywordTok{read_excel}\NormalTok{(}\StringTok{"nla2007_wq.xlsx"}\NormalTok{)}
\end{Highlighting}
\end{Shaded}

This is the simplest case, but lets dig into the options to see what's
possible

\begin{verbatim}
## function (path, sheet = NULL, range = NULL, col_names = TRUE, 
##     col_types = NULL, na = "", trim_ws = TRUE, skip = 0, n_max = Inf, 
##     guess_max = min(1000, n_max), progress = readxl_progress(), 
##     .name_repair = "unique") 
## NULL
\end{verbatim}

\hypertarget{an-aside-on-colum-names}{%
\subsubsection{An aside on colum names}\label{an-aside-on-colum-names}}

If you are new to R and coming from mostly and Excel background, then
you may want to think a bit more about column names than you usually
might. Excel is very flexible when it comes to naming columns and this
certainly has its advantages when the end user of that data is a human.
However, humans don't do data analysis. Computers do. So at some point
the data in that spreadsheet will likely need to be read into software
that can do this analysis. To ease this process it is best to keep
column names simple, without spaces, and without special characters
(e.g.~!, @, \&, \$, etc.). While it is possible to deal with these
cases, it is not straightforward, especially for new users. So, when
working with your data (or other people's data) take a close look at the
column names if you are running into problems reading that data into R.
I suggest using all lower case with separate words indicated by and
underscore. Things like ``chlorophyll\_a'' or ``total\_nitrogen'' are
good examples of decent column names.

\hypertarget{exercise-3.1}{%
\subsection{Exercise 3.1}\label{exercise-3.1}}

For this exercise, let's read in a new dataset but this time, directly
from a URL. We are still working on the \texttt{nla\_analysis.R} Script

\begin{enumerate}
\def\labelenumi{\arabic{enumi}.}
\tightlist
\item
  Add a new line of code, starting after the \texttt{read\_csv} line we
  looked at above (on or around line 39).\\
\item
  Use the \texttt{read\_csv()} function to read in
  ``\url{https://www.epa.gov/sites/production/files/2014-01/nla2007_sampledlakeinformation_20091113.csv}'',
  and assign the output to a data frame named \texttt{nla\_sites}.
\item
  How many rows and columns do we have in this data frame?\\
\item
  What is stored in the fourth column of this data frame?
\end{enumerate}

\hypertarget{tidy-data}{%
\subsection{Tidy data}\label{tidy-data}}

We have learned about data frames, how to create them, and about several
ways to read in external data into a data.frame. At this point there
have been only a few rules applied to our data frames (which already
separates them from spreadsheets) and that is our datasets must be
rectangular. Beyond that we haven't disscussed how best to organize that
data so that subsequent analyses are easier to accomplish. This is, in
my opinion, the biggest decision we make as data analysts and it takes a
lot of time to think about how best to organize data and to actually
re-organize that data. Luckily, we can use an existing concept for this
that will help guide our decisions and re-organization. The best concept
I know of to do this is the concept of
\href{http://r4ds.had.co.nz/tidy-data.html}{tidy data}. The essence of
which can be summed up as:

\begin{enumerate}
\def\labelenumi{\arabic{enumi}.}
\tightlist
\item
  Each column is a variable
\item
  Each row is an observation
\item
  Each cell has a single value
\item
  The data must be rectangular
\end{enumerate}

Lastly, if you want to read more about this there are several good
sources:

\begin{itemize}
\tightlist
\item
  The previously linked R4DS Chapter on
  \href{http://r4ds.had.co.nz/tidy-data.html}{tidy data}
\item
  The \href{https://www.jstatsoft.org/article/view/v059i10}{original
  paper by Hadley Wickham}
\item
  The \href{http://tidyr.tidyverse.org/articles/tidy-data.html}{Tidy
  Data Vignette}
\item
  Really anything on the \href{https://www.tidyverse.org/}{Tidyverse
  page}
\item
  A lot of what is in the
  \href{https://datacarpentry.org/spreadsheet-ecology-lesson/}{Data
  Carpentry Ecology Spreadsheet Lesson} is also very relevant.
\end{itemize}

Let's now see some of the basic tools for tidying data using the
\texttt{tidyr} and \texttt{dplyr} packages.

\hypertarget{data-manipulation-with-dplyr}{%
\subsubsection{\texorpdfstring{Data manipulation with
\texttt{dplyr}}{Data manipulation with dplyr}}\label{data-manipulation-with-dplyr}}

There are a lot of different ways to manipulate data in R, but one that
is part of the core of the Tidyverse is \texttt{dplyr}. In particular,
we are going to look at selecting columns, filtering data, adding new
columns, grouping data, and summarizing data.

\hypertarget{select}{%
\paragraph{select}\label{select}}

Often we get datasets that have many columns or columns that need to be
re-ordered. We can accomplish both of these with \texttt{select}. Here's
a quick example with the \texttt{iris} dataset. We will also be
introducing the concept of the pipe: \texttt{\%\textgreater{}\%} which
we will be using going forward. Let's look at some code that we can
disect.

\begin{Shaded}
\begin{Highlighting}[]
\NormalTok{iris_petals <-}\StringTok{ }\NormalTok{iris }\OperatorTok
\StringTok{  }\KeywordTok{select}\NormalTok{(Species, Petal.Width, Petal.Length) }\OperatorTok
\StringTok{  }\KeywordTok{as_tibble}\NormalTok{() }\CommentTok{#the as_tibble function helps make the output look nice}
\NormalTok{iris_petals}
\end{Highlighting}
\end{Shaded}

\begin{verbatim}
## # A tibble: 150 x 3
##    Species Petal.Width Petal.Length
##    <fct>         <dbl>        <dbl>
##  1 setosa          0.2          1.4
##  2 setosa          0.2          1.4
##  3 setosa          0.2          1.3
##  4 setosa          0.2          1.5
##  5 setosa          0.2          1.4
##  6 setosa          0.4          1.7
##  7 setosa          0.3          1.4
##  8 setosa          0.2          1.5
##  9 setosa          0.2          1.4
## 10 setosa          0.1          1.5
## # ... with 140 more rows
\end{verbatim}

The end result of this is a data frame, \texttt{iris\_petals} that has
three columns: Species, Petal.Width and Petal.Length in the order that
we specified. And the syntax we are now using is ``piped'' in that we
use the \texttt{\%\textgreater{}\%} operator to send something from
before the operator (a.k.a. ``to the left'') to the first argument of
the function after the operator (a.k.a. ``to the right''). This allows
us to write our code in the same order as we think of it. The best
explanation of this is (again) from R For Data Science in the
\href{http://r4ds.had.co.nz/pipes.html}{Piping chapter}.

\hypertarget{filter}{%
\paragraph{filter}\label{filter}}

The \texttt{filter()} function allows us to fiter our data that meets
certain criteria. For instance, we might want to further manipulate our
3 column data frame with only one species of Iris and Petals greater
than the median petal width.

\begin{Shaded}
\begin{Highlighting}[]
\NormalTok{iris_petals_virginica <-}\StringTok{ }\NormalTok{iris }\OperatorTok
\StringTok{  }\KeywordTok{select}\NormalTok{(}\DataTypeTok{species =}\NormalTok{ Species, }\DataTypeTok{petal_width =}\NormalTok{ Petal.Width, }\DataTypeTok{petal_length =}\NormalTok{ Petal.Length) }\OperatorTok
\StringTok{  }\KeywordTok{filter}\NormalTok{(species }\OperatorTok{==}\StringTok{ "virginica"}\NormalTok{) }\OperatorTok
\StringTok{  }\KeywordTok{filter}\NormalTok{(petal_width }\OperatorTok{>=}\StringTok{ }\KeywordTok{median}\NormalTok{(petal_width)) }\OperatorTok
\StringTok{  }\KeywordTok{as_tibble}\NormalTok{()}
\NormalTok{iris_petals_virginica  }
\end{Highlighting}
\end{Shaded}

\begin{verbatim}
## # A tibble: 29 x 3
##    species   petal_width petal_length
##    <fct>           <dbl>        <dbl>
##  1 virginica         2.5          6  
##  2 virginica         2.1          5.9
##  3 virginica         2.2          5.8
##  4 virginica         2.1          6.6
##  5 virginica         2.5          6.1
##  6 virginica         2            5.1
##  7 virginica         2.1          5.5
##  8 virginica         2            5  
##  9 virginica         2.4          5.1
## 10 virginica         2.3          5.3
## # ... with 19 more rows
\end{verbatim}

\hypertarget{mutate}{%
\paragraph{mutate}\label{mutate}}

Now say we have some research that suggest the ratio of the petal width
and petal length is imporant. We might want to add that as a new column
in our data set. The \texttt{mutate} function does this for us.

\begin{Shaded}
\begin{Highlighting}[]
\NormalTok{iris_petals_ratio <-}\StringTok{ }\NormalTok{iris }\OperatorTok
\StringTok{  }\KeywordTok{select}\NormalTok{(}\DataTypeTok{species =}\NormalTok{ Species, }\DataTypeTok{petal_width =}\NormalTok{ Petal.Width, }\DataTypeTok{petal_length =}\NormalTok{ Petal.Length) }\OperatorTok
\StringTok{  }\KeywordTok{mutate}\NormalTok{(}\DataTypeTok{petal_ratio =}\NormalTok{ petal_width}\OperatorTok{/}\NormalTok{petal_length) }\OperatorTok
\StringTok{  }\KeywordTok{as_tibble}\NormalTok{()}
\NormalTok{iris_petals_ratio}
\end{Highlighting}
\end{Shaded}

\begin{verbatim}
## # A tibble: 150 x 4
##    species petal_width petal_length petal_ratio
##    <fct>         <dbl>        <dbl>       <dbl>
##  1 setosa          0.2          1.4      0.143 
##  2 setosa          0.2          1.4      0.143 
##  3 setosa          0.2          1.3      0.154 
##  4 setosa          0.2          1.5      0.133 
##  5 setosa          0.2          1.4      0.143 
##  6 setosa          0.4          1.7      0.235 
##  7 setosa          0.3          1.4      0.214 
##  8 setosa          0.2          1.5      0.133 
##  9 setosa          0.2          1.4      0.143 
## 10 setosa          0.1          1.5      0.0667
## # ... with 140 more rows
\end{verbatim}

\hypertarget{exercise-3.2}{%
\subsection{Exercise 3.2}\label{exercise-3.2}}

For this exercise we will dig into our datasets and find ways to tidy
them up. We first need to clean up the new data
frame,\texttt{nla\_sites}, that we loaded up in Exercise 3.1. Add new
lines of code after the section of code that cleans up the
\texttt{nla\_wq} data frame. Add some comments to your script that
describe what we are doing.

\begin{enumerate}
\def\labelenumi{\arabic{enumi}.}
\tightlist
\item
  Filter out just the first visits (e.g.~VISIT\_NO equal to 1)
\item
  Select the following columns: SITE\_ID, STATE\_NAME, and CNTYNAME
\item
  Make all of our columns names lower case (Hint: Take a look at the
  code in nla\_analysis.R where we manipulate our data)
\item
  Make all the character fields lower case (Hint: Take a look at the
  code in nla\_analysis.R where we manipulate our data)
\item
  Keep all these changes in the \texttt{nla\_sites} data frame.
\end{enumerate}

\hypertarget{group_by-and-summarize}{%
\paragraph{group\_by and summarize}\label{group_by-and-summarize}}

Now back to iris. What if we want to get some summary statistics of our
important petal ratio metric for each of the species? Grouping the data
by species, and then summarizing those groupings will let us accomplish
this.

\begin{Shaded}
\begin{Highlighting}[]
\NormalTok{iris_petal_ratio_species <-}\StringTok{ }\NormalTok{iris }\OperatorTok
\StringTok{  }\KeywordTok{select}\NormalTok{(}\DataTypeTok{species =}\NormalTok{ Species, }\DataTypeTok{petal_width =}\NormalTok{ Petal.Width, }\DataTypeTok{petal_length =}\NormalTok{ Petal.Length) }\OperatorTok
\StringTok{  }\KeywordTok{mutate}\NormalTok{(}\DataTypeTok{petal_ratio =}\NormalTok{ petal_width}\OperatorTok{/}\NormalTok{petal_length) }\OperatorTok
\StringTok{  }\KeywordTok{group_by}\NormalTok{(species) }\OperatorTok
\StringTok{  }\KeywordTok{summarize}\NormalTok{(}\DataTypeTok{mean_petal_ratio =} \KeywordTok{mean}\NormalTok{(petal_ratio),}
            \DataTypeTok{sd_petal_ratio =} \KeywordTok{sd}\NormalTok{(petal_ratio),}
            \DataTypeTok{median_petal_ratio =} \KeywordTok{median}\NormalTok{(petal_ratio))}
\NormalTok{iris_petal_ratio_species}
\end{Highlighting}
\end{Shaded}

\begin{verbatim}
## # A tibble: 3 x 4
##   species    mean_petal_ratio sd_petal_ratio median_petal_ratio
##   <fct>                 <dbl>          <dbl>              <dbl>
## 1 setosa                0.168         0.0658              0.143
## 2 versicolor            0.311         0.0292              0.309
## 3 virginica             0.367         0.0502              0.375
\end{verbatim}

\hypertarget{left_join}{%
\paragraph{left\_join}\label{left_join}}

Lastly, we might also have information spread across multiple data
frames. This is the same concept as having multiple tables in a
relational database. There are MANY ways to combine (aka. "join) tables
like this and most of them have a \texttt{dplyr} verb implemented for
them. We are going to focus on one, the \texttt{left\_join()}.

Instead of continuing with the \texttt{iris} data we will create some
data frames to work with for these examples.

\begin{Shaded}
\begin{Highlighting}[]
\NormalTok{left_table <-}\StringTok{ }\KeywordTok{data.frame}\NormalTok{(}\DataTypeTok{left_id =} \DecValTok{1}\OperatorTok{:}\DecValTok{6}\NormalTok{, }
                         \DataTypeTok{names =} \KeywordTok{c}\NormalTok{(}\StringTok{"Bob"}\NormalTok{, }\StringTok{"Sue"}\NormalTok{, }\StringTok{"Jeff"}\NormalTok{, }\StringTok{"Alice"}\NormalTok{, }\StringTok{"Joe"}\NormalTok{, }\StringTok{"Betty"}\NormalTok{))}
\NormalTok{right_table <-}\StringTok{ }\KeywordTok{data.frame}\NormalTok{(}\DataTypeTok{right_id =} \DecValTok{1}\OperatorTok{:}\DecValTok{5}\NormalTok{, }
                          \DataTypeTok{left_id =} \KeywordTok{c}\NormalTok{(}\DecValTok{2}\NormalTok{,}\DecValTok{1}\NormalTok{,}\DecValTok{3}\NormalTok{,}\DecValTok{6}\NormalTok{,}\DecValTok{7}\NormalTok{), }
                          \DataTypeTok{age =} \KeywordTok{c}\NormalTok{(}\DecValTok{17}\NormalTok{,}\DecValTok{26}\NormalTok{,}\DecValTok{45}\NormalTok{,}\DecValTok{32}\NormalTok{,}\DecValTok{6}\NormalTok{)) }
\NormalTok{left_table}
\end{Highlighting}
\end{Shaded}

\begin{verbatim}
##   left_id names
## 1       1   Bob
## 2       2   Sue
## 3       3  Jeff
## 4       4 Alice
## 5       5   Joe
## 6       6 Betty
\end{verbatim}

\begin{Shaded}
\begin{Highlighting}[]
\NormalTok{right_table}
\end{Highlighting}
\end{Shaded}

\begin{verbatim}
##   right_id left_id age
## 1        1       2  17
## 2        2       1  26
## 3        3       3  45
## 4        4       6  32
## 5        5       7   6
\end{verbatim}

To combine these two tables into one we can \texttt{join} them. In
particular we will use a \texttt{left\_join()} which keeps all records
from the first table (i.e the ``left'' table) and adds only the matching
records in the second table (i.e.~the ``right'' table). This is easier
to grok by looking at the results.

\begin{Shaded}
\begin{Highlighting}[]
\NormalTok{left_right_table <-}\StringTok{ }\NormalTok{left_table }\OperatorTok
\StringTok{  }\KeywordTok{left_join}\NormalTok{(right_table, }\DataTypeTok{by =} \KeywordTok{c}\NormalTok{(}\StringTok{"left_id"}\NormalTok{ =}\StringTok{ "left_id"}\NormalTok{))}
\NormalTok{left_right_table}
\end{Highlighting}
\end{Shaded}

\begin{verbatim}
##   left_id names right_id age
## 1       1   Bob        2  26
## 2       2   Sue        1  17
## 3       3  Jeff        3  45
## 4       4 Alice       NA  NA
## 5       5   Joe       NA  NA
## 6       6 Betty        4  32
\end{verbatim}

\hypertarget{exercise-3.3}{%
\subsection{Exercise 3.3}\label{exercise-3.3}}

Let's now practice combining two data frames and summarizing some
information in that combine data frame. We are still working on the
\texttt{nla\_analysis.R} script and you can add this code after that
section we just completed on \texttt{nla\_sites}. Don't forget your
comments!

\begin{enumerate}
\def\labelenumi{\arabic{enumi}.}
\tightlist
\item
  Use \texttt{left\_join()} to combine \texttt{nla\_wq} and
  \texttt{nla\_sites} into a new data frame called \texttt{nla\_2007}
\item
  Using \texttt{group\_by()} and \texttt{summarize}, let's look at
  median chlorophyll per EPA Region. (Hint: There are some NA's that we
  will need to deal with. Use \texttt{?median} and try to figure out how
  to remove those when doing this calculation)
\item
  Re-do the above and include the minimum and maximum values.
\item
  Bonus: Use `arrange() to order the output by mean chlorophyll.
\end{enumerate}


\end{document}
